\documentclass[10pt]{beamer}



%%%
% PREAMBLE FOR THIS DOC 
%%%
%https://tex.stackexchange.com/questions/68821/is-it-possible-to-create-a-latex-preamble-header
\usepackage{/Users/miw267/Repos/csci246_spring2025/slides/preambles/beamer_preamble_for_CSCI246}

\usepackage{wrapfig}  


%%% TRY TO RESHOW TOC AT EACH SECTION START (with current section highlighted)
% Reference: https://tex.stackexchange.com/questions/280436/how-to-highlight-a-specific-section-in-beamer-toc
\newcommand\tocforsect[2]{%
  \begingroup
  \edef\safesection{\thesection}
  \setcounter{section}{#1}
  \tableofcontents[#2,currentsection]
  \setcounter{section}{\safesection}
  \endgroup
}


\usepackage[normalem]{ulem} % for strikeout (\sout)

%%%% HERES HOW TO DO IT CORRECTLY
% FIRST IN .STY FILE, DO
%\usetheme[sectionpage=none]{metropolis}
% THEN AT EACH SECTION DO
%\begin{frame}{Outline}
%  \tableofcontents[currentsection]	
%\end{frame}



%\setbeamertemplate{navigation symbols}{}
%\setbeamertemplate{footline}[frame number]{}


%%%
% DOCUMENT
%%%

\begin{document}

%\maketitle

%% Title page frame
%\begin{frame}
%    \titlepage 
%\end{frame}



\title{02/10/2026: Markov Chains (Part 1)}
\author{CSCI 546: Diffusion Models}
\date{Textbook reference: Sec 6.1-6.2}

\begin{frame}
    \titlepage 
\end{frame}

\begin{frame}
%\begin{myyellowbox}[title=Opening Discussion ($\approx$ 5 mins)]
%Find a partner and discuss these questions.
%\begin{enumerate}
%	\item What is your early reflection on the class?
%		\begin{itemize}
%		\item[a)] goal
%		\item[b)] hope
%		\item[c)] fear
%		\item[d)] like
%		\item[e)] dislike
%		\end{itemize}
%	\item Reflect on your experience reading the textbook.
%\end{enumerate}
%\end{myyellowbox}
%\vfill 
%\pause 
\begin{mygreenbox}[title=Announcement (Sign-in Sheet)]
Please sign the sign-in sheet.
\end{mygreenbox}
%\vfill 
%\pause 
%\begin{myredbox}[title=\text{Announcement (Office Hours)}]
%My office hours today need to be changed due to a grant meeting.  If you would like to meet on today, please send me an email to set up a time.
%\end{myredbox}
%\vfill 
%\pause 
%\begin{myyellowbox}[title=\text{Announcement (Group exercises)}]
%Group exercises will be posted to the course repo after class. Please continue working on what you don't finish in class.  
%\end{myyellowbox}
\end{frame}

\begin{frame}[standout]
Review Problem Set \#7	
\end{frame}

\begin{frame}[standout]
Concepts for Problem Set \#8
\end{frame}

\begin{frame}{Markov Chains}

Suppose we observe a random sequence $(x_i)_{i=1}^\infty$.  


By chain rule, we can \alert{always} write the joint density over the first $n$ observations as:
\begin{align*}
p(x_1,x_2,\hdots, x_n) &= p(x_1) p(x_2 \mid x_1) p(x_3 \mid x_1, x_2) \cdots p(x_n \mid x_1, x_2, \hdots x_{n-2}, x_{n-1})	
\end{align*}
\pause 
\red{Markov chains} are special sequences whose joint density simplifies to:
\begin{align*}
p(x_1,x_2,\hdots, x_n) &= p(x_1) p(x_2 \mid x_1) p(x_3 \mid \red{\cancel{x_1}}, x_2) \cdots p(x_n \mid  \red{\cancel{x_1, x_2}}, \hdots  \red{\cancel{x_{n-2}}}, x_{n-1})	
\end{align*}
\pause 
That is, Markov chains satisfy
\begin{align*}
p(x_{n+1} \cond x_{1:n}) = p(x_{n+1} \mid x_{n})
\end{align*}
\pause 
\vfill 
In words:
\vspace{.2cm}

\fcolorbox{red!60}{red!15}{
  \parbox{0.9\linewidth}{
    The future is conditionally independent of the past given the present.
  }
}

\end{frame}


\begin{frame}{Example: Simple random walk}

For timesteps $n=1,2,3,\hdots$, let 
\[ X_n =
\begin{cases}
1, &\text{ with probability } p \\
-1, &\text{ with probability } q=1-p
\end{cases}
 \]

Then a \alert{simple random walk} (in one dimension) is given by
\[ S_n = S_0 + \sum_{i=1}^n X_i,\]
where $S_0$ is the starting position.

\vfill 

\begin{center}
\includegraphics[width=.7\textwidth]{images/random_walk.png}
\end{center}	

\end{frame}


\begin{frame}{Random Groups}

\begin{columns}
\begin{column}{0.33\textwidth}
Aubrey Williams: 6 \\ 
Austin Barton : 6 \\ 
Blake Sigmundstad: 3 \\ 
Diego Moylan: 2 \\ 
Dillon Shaffer: 1 \\ 
Ismoiljon Muzaffarov: 7 \\ 
Jacob Tanner: 3 \\\end{column}
\begin{column}{0.33\textwidth}
Josh Stoneback: 3 \\ 
Joshua Bowen: 1 \\ 
Joshua Culwell: 5 \\ 
Laura Banaszewski: 5 \\ 
Lina Hammel: 4 \\ 
Logan Racz: 1 \\\end{column}
\begin{column}{0.33\textwidth}
Matt Hall: 4 \\ 
Micah Miller: 8 \\ 
Mike Kadoshnikov: 8 \\ 
Owen Cool: 2 \\ 
Racquel Bowen: 2 \\ 
Samuel Mocabee: 4 \\ 
Tatiana Kirillova: 7 \\\end{column}
\end{columns}

\end{frame}


\begin{frame}{Group exercises - Problem Set 8} 
\begin{enumerate}
\item (6.1.2) A die is rolled repeatedly.  Which of the following are Markov chains? For those that are, supply the transition matrix
	\begin{itemize}
	\item[(a)] The largest number $X_n$ shown up to the $n$-th roll.
	\item[(b)] The number $N_n$ of sixes in $n$ rolls.
	\item[(c)] At time $r$, the time $C_r$ since the most recent six.
	\item[(d)] At time $r$, the time $B_r$ until the next six.
	\end{itemize}
\item (Sec 6.2) Show that a simple random walk is periodic with period 2.  Then show that a simple random walk is recurrent if $p=\half$ but transient if $p \neq \half$.  \pause 
	\begin{itemize}
	\item Hint: Use Stirling's formula: $n! \sim \sqrt{2 \pi n} \, \big(\frac{n}{e}\big)^n$. \pause 
	\item For additional help, see slide 22 \href{https://www.stat.uchicago.edu/~yibi/teaching/stat317/2021/Lectures/Lecture3.pdf}{\blue{\underline{here.}}} 
	\end{itemize}
	

\end{enumerate}   
\end{frame}


\end{document}
