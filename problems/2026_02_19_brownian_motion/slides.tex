\documentclass[10pt]{beamer}



%%%
% PREAMBLE FOR THIS DOC 
%%%
%https://tex.stackexchange.com/questions/68821/is-it-possible-to-create-a-latex-preamble-header
\usepackage{/Users/miw267/Repos/csci246_spring2025/slides/preambles/beamer_preamble_for_CSCI246}

\usepackage{wrapfig}  


%%% TRY TO RESHOW TOC AT EACH SECTION START (with current section highlighted)
% Reference: https://tex.stackexchange.com/questions/280436/how-to-highlight-a-specific-section-in-beamer-toc
\newcommand\tocforsect[2]{%
  \begingroup
  \edef\safesection{\thesection}
  \setcounter{section}{#1}
  \tableofcontents[#2,currentsection]
  \setcounter{section}{\safesection}
  \endgroup
}


\usepackage[normalem]{ulem} % for strikeout (\sout)

%%%% HERES HOW TO DO IT CORRECTLY
% FIRST IN .STY FILE, DO
%\usetheme[sectionpage=none]{metropolis}
% THEN AT EACH SECTION DO
%\begin{frame}{Outline}
%  \tableofcontents[currentsection]	
%\end{frame}



%\setbeamertemplate{navigation symbols}{}
%\setbeamertemplate{footline}[frame number]{}


%%%
% DOCUMENT
%%%

\begin{document}

%\maketitle

%% Title page frame
%\begin{frame}
%    \titlepage 
%\end{frame}



\title{02/19/2026: Brownian Motion}
\author{CSCI 546: Diffusion Models}
\date{Textbook reference: Chang Sec 5.1-5.6, GS Sec  8.1, 8.5}

\begin{frame}
    \titlepage 
\end{frame}

\begin{frame}

\begin{mygreenbox}[title=Announcement (Sign-in Sheet)]
Please sign the sign-in sheet.
\end{mygreenbox}

\vfill 
\pause 
 
\begin{myredbox}[title=\text{Personal Note}]
\begin{center}
\includegraphics[width=.5\textwidth]{images/compton.png}	
\end{center}

\end{myredbox}

\end{frame}

\begin{frame}[standout]
Review Problem Set \#10
\end{frame}

\begin{frame}[standout]
Concepts for Problem Set \#11
\end{frame}

%
%\begin{frame}[standout]
%Outline for today's material
%\begin{itemize}
%\item \textbullet \quad \alert{Reversibility}
%\item \textbullet \quad Chains with finitely many states
%\item \textbullet \quad Continuous-time Markov chains
%\end{itemize}
%
%\end{frame}


\begin{frame}{Stochastic Process}

\begin{center}
\includegraphics[width=.67\textwidth]{images/brownian_motion_paths.png}
\end{center}	

\colorbox{blue!30}{\textbf{Definition.}} A \textbf{stochastic process} $\set{X(t): t \in \R}$ is a function $X: \R \times \Omega \to \R$. Thus, we can view $X=X(t,\omega)$ in two ways:
\begin{itemize}
\item \textit{As a collection of random variables.} For each fixed $t$, $X(t, \cdot)$ is a random variable.
\item \textit{As a random function.} For each fixed $\omega$, $X(\cdot, \omega)$ is a real-valued function (or ``path'').
\end{itemize} 

\end{frame}



\begin{frame}{Brownian Motion}

\begin{center}
\includegraphics[width=.6\textwidth]{images/brownian_motion_paths.png}
\end{center}	

\colorbox{blue!30}{\textbf{Definition.}} A \textbf{standard Brownian motion} $\set{B(t): t \geq 0}$ is a stochastic process having
\begin{itemize}
\item[(i)] continuous paths,
\item[(ii)] stationary, independent increments, and
\item[(iii)] $B(t) \sim N(0,t)$ for all $t \geq 0$.
\end{itemize} 

\only<2>{\colorbox{yellow!30}{\textbf{Poll.}} What does ``stationary, independent increments'' mean?}

\only<3-4>{\colorbox{yellow!30}{\textbf{Remark.}} A general  $(\mu, \sigma^2)$  Brownian motion has mean and variance increasing at rate $\mu$ and $\sigma^2$ per unit time:} \only<4>{$X(t)=X(0)+\mu t + \sigma B(t).$}

\end{frame}

\begin{frame}{Is Brownian Motion a Gaussian Process?}
\small 
\colorbox{blue!30}{\textbf{Definition.}} A \textbf{Gaussian process} $\set{X(t): t \geq 0}$ is a stochastic process such that for all numbers $n=1,2,\hdots$ and all times $t_1, t_2, \hdots t_n$, the random vector $\big( X(t_1), X(t_2), \hdots, X(t_n) \big)$ has a joint normal distribution.
\vfill 
\only<2-7>{\colorbox{yellow!30}{\textbf{Poll.}} Is Brownian motion a Gaussian process? Why?}
\vfill 
\begin{columns}[T]  % T = align at top

% ---------- LEFT: TEXT ----------
\begin{column}{0.7\textwidth}

\only<3-6>{
\colorbox{red!30}{\textbf{Warning.}}
\textbf{Gaussian marginals do not imply a jointly Gaussian distribution.}
} \only<4-6>{For instance, let
\begin{itemize}
  \item $Z \sim \mathcal N(0,1)$,
  \item $\varepsilon$ be independent of $Z$, with
        $P(\varepsilon=1)=P(\varepsilon=-1)=\half$,
  \item $X := Z, \qquad Y := \varepsilon Z$.
\end{itemize}
}

\only<5-6>{
\textbf{Both marginals are Gaussian:}
\begin{itemize}
  \item $X \sim \mathcal N(0,1)$,
  \item Since $\varepsilon$ just flips the sign,
        $Y \stackrel{d}{=} Z \sim \mathcal N(0,1)$.
\end{itemize}
}

\only<6>{
\textbf{But the joint is not Gaussian:}
\begin{itemize}
  \item $(X,Y)$ lives on the two lines $y=x$ and $y=-x$.
\end{itemize}
}

\end{column}

% ---------- RIGHT: PICTURE ----------
\begin{column}{0.45\textwidth}
\only<4-6>{
\centering
\includegraphics[width=\textwidth]{images/gaussian_marginals_doesnt_imply_jointly_gaussian.png}
}
\end{column}

\end{columns}

\only<7-8>{
\colorbox{green!30}{\textbf{Solution.}} Yes!
}

\only<8->{
Sketch of argument:
\begin{itemize}
  \item<8-> The increments $\Delta_i = X(t_i) - X(t_{i-1})$ are all Gaussian.
  \item<9-> Since the $\Delta_i$’s are also independent, the vector of increments
  \[
    \Delta := (\Delta_1, \ldots, \Delta_n)
  \]
  is jointly Gaussian.
  \item<10-> We can write
  \[
    (X(t_1), X(t_2), \ldots, X(t_n)) = A \Delta
  \]
  for some matrix $A$, i.e.\ as a linear transformation of $\Delta$.
  Linear transformations of multivariate Gaussians are multivariate Gaussians.
\end{itemize}
}
\end{frame}

\begin{frame}{Gaussian Processes}
\begin{center}
\includegraphics[width=\textwidth]{images/gps_with_different_covariances.png} \\
\textbf{Different covariance functions give different types of GPs.}
\end{center}

\end{frame}


\begin{frame}{Brownian Motion as a Gaussian Process}

\colorbox{green!30}{\textbf{Fact (from Chang).}} A Gaussian process having continuous paths, mean 0, and covariance function $r(s,t)=s \wedge t$ is a standard Brownian motion.
\vfill
\pause 
\colorbox{red!30}{\textbf{Notation.}} $s \wedge t = \min(s,t)$.
\end{frame}


\begin{frame}{Random Groups}

\begin{columns}
\begin{column}{0.33\textwidth}
Aubrey Williams: 6 \\ 
Austin Barton : 2 \\ 
Blake Sigmundstad: 4 \\ 
Diego Moylan: 5 \\ 
Dillon Shaffer: 3 \\ 
Ismoiljon Muzaffarov: 3 \\ 
Jacob Tanner: 1 \\\end{column}
\begin{column}{0.33\textwidth}
Josh Stoneback: 1 \\ 
Joshua Bowen: 1 \\ 
Joshua Culwell: 5 \\ 
Laura Banaszewski: 3 \\ 
Lina Hammel: 4 \\ 
Logan Racz: 6 \\\end{column}
\begin{column}{0.33\textwidth}
Matt Hall: 2 \\ 
Micah Miller: 4 \\ 
Mike Kadoshnikov: 7 \\ 
Owen Cool: 2 \\ 
Racquel Bowen: 8 \\ 
Samuel Mocabee: 8 \\ 
Tatiana Kirillova: 7 \\\end{column}
\end{columns}
\end{frame}



\begin{frame}{Group exercises - Problem Set 11}
\begin{enumerate}
\item (Chang Exercise 5.5) \textbf{Brownian Motion keeps ``restarting'', and is Markov.} Suppose that $W$ is standard Brownian motion, and let $c>0$.  Define $X(t)=W(c+t)-W(c)$. Then $\set{X(t): t \geq 0}$ is a standard Brownian motion that is independent of $\set{W(t): 0 \leq t \leq c}$.
\item (Chang Sec 5.1) \textbf{Covariance of Brownian Motion.}  Prove that for Brownian motion, $\Cov(W_s, W_t) = s \wedge t.$ (Recall the notation that $s \wedge t := \min(s,t)$).
\item (GS 8.6.5) Let $W$ be standard Brownian motion. Which of the following are also standard Brownian motions?
\[ (a) \, -W(t), \qquad (b) \, \sqrt{t} W(1), \qquad (c) \, t W(1/t).\]
\end{enumerate}   
\end{frame}


\end{document}
