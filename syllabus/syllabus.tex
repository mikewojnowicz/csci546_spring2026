

%%%%%%%%%%%%%%%%%%%%%%%%%%%%%%%%%%%%%%%%%
% Inzane Syllabus Template
% LaTeX Template
% Version 1.2 (8.22.2019)
%
% This template has been downloaded from:
% http://www.LaTeXTemplates.com
%
% Original author:
% Carmine Spagnuolo (cspagnuolo@unisa.it) with major modifications by 
% Zane Wolf (zwolf.mlxvi@gmail.com)
%
% I (Zane) have left a lot of instructions both in the .tex file and the .cls file that can guide you to customize this document to suite your tastes and requirements. Here is a brief guide: 
%  - Changing the Main Color: .cls line 39
%  - Adding more FAQs: .cls line 126 and .tex line 99
%  - Adding TA emails: uncomment .cls lines 220 & 224 and .tex lines 85 and 89
%  - Deleting the FAQ sidebar entirely: .tex line 188
%  - Removing the Lab/TA Info and placing a brief Overview/About section in their place:        uncomment .tex line 91 and .cls line 227, and comment .cls lines for the LAB/TA info        that you no longer want (c. lines 184-227)

%
% I am also happy to help with crafting/designing modifications to this template to help suite your personal needs in a syllabus. Feel free to reach out! 
%
% License:
% The MIT License (see included LICENSE file)
%
%%%%%%%%%%%%%%%%%%%%%%%%%%%%%%%%%%%%%%%%%

%----------------------------------------------------------------------------------------
%	PACKAGES AND OTHER DOCUMENT CONFIGURATIONS
%----------------------------------------------------------------------------------------

\documentclass[letterpaper]{inzane_syllabus} % a4paper for A4

\usepackage{preamble_syllabus}


\usepackage{booktabs, colortbl, xcolor}
\usepackage{tabularx}
\usepackage{enumitem}
\usepackage{ltablex} 
\usepackage{multirow}

\setlist{nolistsep}

\usepackage{lscape}
\newcolumntype{r}{>{\hsize=0.9\hsize}X}
\newcolumntype{w}{>{\hsize=0.6\hsize}X}
\newcolumntype{m}{>{\hsize=.9\hsize}X}

\renewcommand{\familydefault}{\sfdefault}
\renewcommand{\arraystretch}{2.0}
%----------------------------------------------------------------------------------------
%	 PERSONAL INFORMATION
%----------------------------------------------------------------------------------------

\profilepic{images/rock_stack.png} % Profile picture, if the height of the picture is less than that of the cirle, it will have a flat bottom. 

% To remove any of the following, you need to comment/delete the lines in the .cls file (c. line 186). Commenting/deleting the lines below will produce an error. 

%To add different lines, you will need to create the new command, e.g. \profPhone, in the .cls file c. line 76, and command to create the line in the side bar in the .cls file c. line 186

\classname{Advanced \\ Artificial \\ Intelligence: \\Diffusion 
Models} 
\classnum{CSCI 546} 

%%%%%%%%%%%%%%% PROF INFO
\profname{Dr. Michael  Wojnowicz (Mike)}
\officehours{Office Hrs: Tues 12:15-1:15, Thurs 12-2} 
\office{Barnard 352}
\site{https://mikewojnowicz.github.io/} 
\email{michael.wojnowicz@montana.edu}

%%%%%%%%%%%%%%% COURSE  INFO
\prereq{Prerequisites: Ability to code in some language (ideally Python).  Integral and Differential Calculus.  Previous experience with or strong interest in probability.}
\classdays{}
\classhours{Class meetings: Tues, Thurs 10:50-12:05p}
\classloc{Wilson 1123}



%%%%%%%%%%%%%%% TA INFO
\taAname{}
\taAofficehours{}
\taAoffice{}
\taAemail{}
\taBname{}
\taBofficehours{}
\taBoffice{}
\taBemail{}


%%%%%%%%%%%%%%% TUTOR
\tutorname{}
\tutoremail{}
\tutorofficehours{}
\tutoroffice{}

%%%% FAQs

%to add more questions or remove this section, go to the .cls file and start with lines comment
%lines 226-250. Also comment out this section as well as line 152(ish), the command \makeSide

\qOne{What is a probability distribution?}
\aOne{A discrete structure refers to a mathematical system that is composed of distinct, separate  elements, as opposed to continuous structures where elements can vary smoothly. Think of a digital clock  vs. an analog clock (where second hand loops around continuously without stopping). Examples of discrete structures include sets with finitely many elements (e.g. the integers 1 to 10), lists, graphs, and logical statements. }


\qTwo{What is discrete mathematics?}
\aTwo{Discrete mathematics is the study of discrete structures and mathematical operations that can be performed upon them.}


\qThree{Why study discrete mathematics?}
\aThree{%It does not directly
%help us write programs. At the same time, 
It is the mathematics underlying almost all of computer science. Here are a few examples: (1)  designing high-speed networks and message routing paths,  (2) finding good algorithms for sorting, (3) performing web searches, (4) analyzing algorithms for correctness and efficiency, (5) formalizing security requirements, and (6) designing cryptographic protocols.
}

\qFour{Where is discrete mathematics used in the MSU computer science curriculum?}
\aFour{Discrete mathematics is used  throughout the curriculum.  Proofs in particular play a critical role in CSCI 338 (Computer Science Theory) and CSCI 432 (Advanced Algorithm Topics).
}



%----------------------------------------------------------------------------------------

\begin{document}


\makeprofile % Print the sidebar

%%%%

\section{Course description}

This course provides an in-depth study of diffusion-based generative models in machine learning, including score-based models, denoising diffusion probabilistic models (DDPMs), and conditional and spatiotemporal variants. Students will explore the theoretical foundations, including Markov chains, stochastic differential equations, and score matching.   Students will also learn how transformers can be used within diffusion modeling.  Practical aspects of diffusion modeling will be explored in a final project.

%%%%%
\vspace{0.2cm} 
\section{Readings}

\begin{itemize}
	\item \textit{First half of the course}: Here we will focus on mathematics behind diffusion modeling.  We will use the textbook:
\begin{quotation}
Grimmett, G., \& Stirzaker, D. (2020). \textit{Probability and random processes.} (4th Edition). Oxford university press.	
\end{quotation}
\item \textit{Second half of the course:} Here we will focus on the machine learning of diffusion modeling, and will use the various resources given in the Class Schedule.
\end{itemize}





%%%%%
\vspace{0.2cm}
\section{Learning Outcomes}

%use \begin{outline} or \begin{outline}[enumerate] to create a list with subitems. 
\begin{itemize}
\item To develop fluency in the mathematics underlying diffusion modeling (probability, Markov chains, and diffusion processes).
\item To understand the fundamentals of diffusion modeling in machine learning;
\item To apply diffusion modeling to a real or simulated dataset.
\end{itemize}


%{\color{myCOLOR} Other}\\
%Any required journal articles and book chapters will be provided on \red{Canvas}. 


%%%%%
\vspace{0.2cm} 
\section{Learning Philosophy}

This course will use an ``inquiry-based learning" approach. This means that classes will not be lecture-oriented. Instead, to quote Dr. Dana Ernst,

\begin{quotation}
\textit{You will be expected to work actively to construct your own understanding of the topics at hand with the readily available help of me and your classmates. Many of the concepts you learn and problems you work on will be new to you and ask you to stretch your thinking. You will experience frustration and failure before you experience understanding. This is part of the normal learning process. If you are doing things well, you should be confused at different points in the semester.}
\end{quotation}

Active learning has been shown to increase student performance in STEM courses (e.g. see \cite{deslauriers2011improved, freeman2014active}). 

%%%%
\vspace{0.2cm}
\section{Class activity}

A typical class meeting will be  structured (approximately) as follows:

\noindent
\begin{minipage}{0.45\textwidth}
Mathematics Module
\begin{itemize}
\item Review prev. exercises: 30 mins
\item Mini-lecture: 15 mins
\item New group exercises: 30 mins	
\end{itemize}
\end{minipage}
\hfill
\begin{minipage}{0.45\textwidth}
ML Module
\begin{itemize}
\item Reaction card: 5 mins
\item Peer lightning Pres: 10 mins
\item Paper discussion: 60 mins
\end{itemize}
\end{minipage}

%The course structure follows best practices developed by the Carl Wieman Science Education Initiative (CWSEI).

%%%%%
\vspace{0.1cm} 

%%%%%
{\color{myCOLOR} Daily readings}

Before each class meeting, you will be assigned a reading (or task). These tasks are listed on the class schedule at the end of the syllabus. The reading serves as preparation for actively participating in the class that day.  %Before some classes, you will be given a short reader response questionnaire (with questions like, "what did you find confusing?") posted as a survey Brightspace.  Each class meeting will begin with a 5-minute reading quiz.


{\color{myCOLOR} Group exercises}

For each class meeting within the mathematics module, students will be split randomly into groups of three to work on problems using the whiteboards. Mathematics is not a spectator sport!
%The group exercises are for collaborative problem-solving.


%%%%%
%{\color{myCOLOR} Weekly quiz}
%
%The Friday weekly quizzes will cover group exercises and the readings from the most recent three class meetings (i.e., Monday, Wednesday, and the previous Friday). 

% Lecture is to go over the reading quiz, cover things people didn't understand from the current/previous reading. 

%%%%
%%%%% NEW PAGE 

\newpage % Start a new page

%\makeSide % Print the FAQ sidebar; To get rid of, simply comment out and uncomment \makeFullPage

\makeFullPage

\section{Grading}

\begin{itemize}
\item Midterm: 30\% 
\item Participation: 10\%  
\item Reading reactions: 20\%
\item Presentations: 15\%
\item Project: 25\% 
\end{itemize}

Grades will be assigned as follows: \\
A: 93-100, A-: 90-93, B+: 87-90, B: 83-87, B-: 80-83, C+: 77-80, C: 73-77, C-: 70-73, D+: 67-70, D: 63-67, D-: 60-63, F: 0-60.


%%%%% 
%\vspace{0.4cm}
%\section{Rubric} Quizzes will be graded according to the following rubric.
%\begin{itemize}
%\item (4 points) Correct and well-written.
%\item  (3 points) Good work but some mathematical or writing errors that need addressing.
%\item (2 points) Some good intuition, but there is at least one serious flaw.
%\item (1 point) I don't understand this, but I see that you did work on it.
%\item (0 points) No work is evident.
%\end{itemize}


%%%%
%\vspace{0.3cm}
%\section{Project.}  \red{TODO} The purpose of the project is for students to focus on some aspect of discrete mathematics of particular interest.  Students will work in small groups, choose some aspect of discrete mathematics (that was not covered in class), and develop a 5 minute presentation to give to the class.   For example, a group might give an overview to public key cryptography, or explore how graph theory underlies web search.   There will also be a written component. More information will be given as the time approaches.


\vspace{0.3cm}
\section{Makeup policy.}  It is highly recommended to keep up with the class, as the material is cumulative.  However, I drop 2 participation days with no consequence. Missed material beyond that (with a valid explanation and documentation) can be replaced with by increasing the weight of the midterm and final project grades. 

\vspace{0.3cm}
%\section{Final Exam}
%
%The final will be 2pm-3:50pm on Wednesday, May 7 in Reid 401.
%
%\vspace{0.4cm}
%\section{Communication expectations}
%
%For non-personal questions related to course content, please use the Ask Your Instructor form on D2L, so that all students can benefit from the question and answer.   For personal questions, please contact me by e-mail.

%\newpage % Start a new page
%\makeEmptySide

\vspace{0.3cm}
\section{Diversity and Inclusivity Statement.} I consider this classroom to be a place where you will be treated with respect, and I welcome individuals of all ages, backgrounds, beliefs, ethnicities, genders, gender identities, gender expressions, national origins, religious affiliations, sexual orientations, ability - and other visible and non-visible differences. All members of this class are expected to contribute to a respectful, welcoming and inclusive environment for every other member of the class.  Your suggestions about how to improve the value of diversity in this course are encouraged and appreciated. Please let me know ways to improve the effectiveness of the course for you personally or for other students or student groups.
 
\vspace{0.3cm}
\section{Accommodations for Students with Disabilities.} If you are a student with a disability and wish to use your approved accommodations for this course, contact me during my office hours to discuss. Please have your Accommodation Notification available for verification of accommodations. Accommodations are approved through the Office of Disability Services located in 137 Romney Hall.  www.montana.edu/disabilityservices.

\vspace{0.3cm}
\section{Student Conduct.} You are expected to abide by MSU's Code of Student Conduct.
%
%\vspace{0.4cm}
%\section{Land Acknowledgement}
%
%Living in Montana, we are on the ancestral lands of American Indians, including
%the 12 tribal nations that call Montana home today: A’aninin (Gros Ventre),
%Amskapi/Piikani (Blackfeet), Annishinabe (Chippewa/Ojibway), Annishinabe/Métis
%(Little Shell Chippewa), Apsáalooke (Crow), Ktunaxa/Ksanka (Kootenai), Lakota,
%Dakota (Sioux), Nakoda (Assiniboine), Ne-i-yah-wahk (Plains Cree), Qíispé (Pend
%d’Oreille), Seliš (Salish), and Tsétsêhéstâhese/So’taahe (Northern Cheyenne).
%We honor and respect these tribal nations as we live, work, learn, and play in
%this state. To learn more about Montana Indians, consider reading the pamphlet \textit{Essential Understandings Regarding Montana Indians}, available online.


\vspace{0.3cm}
\section{Scholarly Responsibilities.} The class has no homework. However, to be successful in the course, you need to do the following:

\begin{itemize}
	\item \textbf{Before class:} \textit{Read the assigned section.} Be sure to read \alert{actively}, with a pencil and paper in hand.  For the mathematics module, work out some of the examples within the section on your own, using the textbook to check your work.  %You do NOT need to do the textbook exercises (although you may).
	\item \textbf{During class:} \textit{Attend class.} This allows you to participate in the group activities and discussions.
	\item \textbf{After class:} \textit{Finish working through the group exercises.}   Solutions to group exercises are posted in Canvas.  During class time, you will probably just get the gist of some problems. After class, you should make sure you know how to do \alert{all} of the group exercises.
%	\begin{enumerate}
%	\item Make sure the \alert{form} of your solution matches the provided solutions (e.g. in terms of amount of justification).  
%	\item Make sure you know how to do \alert{all} the group exercises.   
%	\end{enumerate}	
\end{itemize}	


%%%%%%%%%%%%%%%%%%%%%%%%%%%%%%%%%%%%%%%%%%%%%%%%%%%%%%%%%%%%%%%%%%%%%%%%%%%%%
%                COURSE SSec  EDULE
%%%%%%%%%%%%%%%%%%%%%%%%%%%%%%%%%%%%%%%%%%%%%%%%%%%%%%%%%%%%%%%%%%%%%%%%%%%%%
\newpage
\makeFullPage
\section{Class Schedule}

%https://github.com/mikewojnowicz/csci246_spring2025/blob/main/LINKS.md

\begin{center}
\begin{tabularx}{\textwidth}{p{2cm}p{2cm}p{8cm}p{8cm}} %change the width of the comments by changing these cm measurements. Add/substract columns by adding/deleting p{} sections. 
\arrayrulecolor{myCOLOR}
\multicolumn{4}{l}{\textbf{\textcolor{myCOLOR}{\huge MODULE 0: Course Overview}}} \\
\hline \hline 
Tues & Jan 13 & Course Overview & \\
&&& \\
%%%%%%%%%%%%%%%%%%%%%%%%%%%%%%%%%%%%
%%%%% MODULE 1
\arrayrulecolor{myCOLOR}
\multicolumn{4}{l}{\textbf{\textcolor{myCOLOR}{\huge MODULE 1: Mathematics}}} \\
\hline \hline 
&&& \\
\multicolumn{4}{l}{\textbf{\textcolor{myCOLOR}{\large MODULE 1A: Probability Fundamentals}}} \\
\hline
Thurs & Jan 15 &  Events and their probabilities & GS Sec  1.1-1.7\\
Tues & Jan 20 &  Random variables and their distributions & GS Sec  2.1-2.3 \\
Thurs & Jan 22 &  Random variables and their distributions & GS Sec  2.4-2.6 \\
Tues & Jan 27 & Discrete random variables & GS Sec  3.1-3.5 \\
Thurs & Jan 29 & Discrete random variables & GS Sec  3.6-3.10 \\
Tues & Feb 3 & Continuous random variables & GS Sec  4.1-4.5 \\
Thurs & Feb 5 & Continuous random variables & GS Sec  4.6-4.11 \\
\hline
\multicolumn{4}{l}{\textbf{\textcolor{myCOLOR}{\large MODULE 1B: Markov Chains}}} \\
\hline

Tues & Feb 10 & Markov Chains & GS Sec  6.1, 6.2 \\
Thurs & Feb 12 & Markov Chains & GS Sec  6.3, 6.4 \\
Tues & Feb 17 & Markov Chains & GS  Sec  6.5, 6.6, 6.9  \\
\hline
\multicolumn{4}{l}{\textbf{\textcolor{myCOLOR}{\large MODULE 1C: Diffusion Processes}}} \\
\hline
Thurs & Feb 19 & Introduction to diffusion processes & GS Sec  8.1, 8.5, 13.1, 13.2 \\
Tues & Feb 24 & Diffusion Processes & GS  Sec  13.3, 13.7 \\
Thurs & Feb 26 & Diffusion Processes & GS Sec  13.8, 13.9 \\
\hline
\multicolumn{4}{l}{\textbf{\textcolor{myCOLOR}{\large MIDTERM}}} \\
\hline
Tues & Mar 3 & Review for Midterm & \\
Thurs & Mar 5 & MIDTERM EXAM & \\

&&& \\
\arrayrulecolor{myCOLOR}
\multicolumn{4}{l}{\textbf{\textcolor{myCOLOR}{\huge MODULE 2: Machine Learning}}} \\
\hline \hline 
&&& \\
\multicolumn{4}{l}{\textbf{\textcolor{myCOLOR}{\large MODULE 2A: Introduction to Diffusion}}} \\
\hline
Tues & Mar 10 & Variational Inference & Blei, D. M., Kucukelbir, A., \& McAuliffe, J. D. (2017). Variational inference: A review for statisticians. \textit{Journal of the American statistical Association}, 112(518), 859-877.\\
Thurs & Mar 12 & Variational Autoencoders & Kingma, D. P., \& Welling, M. (2014). Auto-encoding variational Bayes. In \textit{Proceedings of the International Conference on Learning Representations (ICLR)}. \\
&  & No Classes --- Spring Break & \\
Tues & Mar 24 & Introduction to Diffusion & Sohl-Dickstein, J., Weiss, E., Maheswaranathan, N., \& Ganguli, S. (2015, June). Deep unsupervised learning using nonequilibrium thermodynamics. In \textit{International conference on machine learning} (pp. 2256-2265).\\
\hline
\multicolumn{4}{l}{\textbf{\textcolor{myCOLOR}{\large MODULE 2B: Diffusion and Score Matching}}} \\
\hline
Thurs & Mar 26 & Score Matching &  Hyvärinen, A., \& Dayan, P. (2005). Estimation of non-normalized statistical models by score matching. \textit{Journal of Machine Learning Research}, 6(4).\\
Tues & Mar 31 & Diffusion Modeling with Score Matching & Ho, J., Jain, A., \& Abbeel, P. (2020). Denoising diffusion probabilistic models. \textit{Advances in neural information processing systems}, 33, 6840-6851. \\
%Thurs & Apr 2 & Tweedie's Formula &  Efron, B. (2011). Tweedie’s formula and selection bias. \textit{Journal of the American Statistical Association}, 106(496), 1602-1614.\\
Thurs & Apr 2 & Diffusion and variational inference & Kingma, D., \& Gao, R. (2023). Understanding diffusion objectives as the elbo with simple data augmentation. \textit{Advances in Neural Information Processing Systems}, 36, 65484-65516 \\
Tues & Apr 7 & Diffusion and SDEs & Song, Y., Sohl-Dickstein, J., Kingma, D. P., Kumar, A., Ermon, S., \& Poole, B. (2021). Score-based generative modeling through stochastic differential equations. In \textit{Proceedings of the International Conference on Learning Representations (ICLR).} \\
\hline
\multicolumn{4}{l}{\textbf{\textcolor{myCOLOR}{\large MODULE 2C: Diffusion with Transformers}}} \\
\hline
Thurs & Apr 9  & Intro to Transformers  & Sasha Rush's \textit{LLMs in 5 formulas}  \\ 
Tues & Apr 14 &  Attention Exercises (Part 1) & \url{https://classic.d2l.ai/chapter_attention-mechanisms/index.html} \\
Thurs & Apr 16 &  Attention Exercises (Part 2) & \url{https://classic.d2l.ai/chapter_attention-mechanisms/index.html} \\
Tues &  Apr 21  &  Diffusion with transformers & Peebles, W., \& Xie, S. (2023). Scalable diffusion models with transformers. In \textit{Proceedings of the IEEE/CVF international conference on computer vision} (pp. 4195-4205). \\
\hline
\multicolumn{4}{l}{\textbf{\textcolor{myCOLOR}{\large MODULE 2D: Spatiotemporal Diffusion}}} \\
\hline
Thurs &  Apr 23 & Conditional score-based diffusion & Tashiro, Y., Song, J., Song, Y., \& Ermon, S. (2021). Csdi: Conditional score-based diffusion models for probabilistic time series imputation. \textit{Advances in neural information processing systems}, 34, 24804-24816. \\
Tues &  Apr 28 & Probabilistic weather forecasting  & Price, I., Sanchez-Gonzalez, A., Alet, F., Andersson, T. R., El-Kadi, A., Masters, D.... \& Willson, M. (2025). Probabilistic weather forecasting with machine learning. \textit{Nature}, 637(8044), 84-90. \\
\hline
\multicolumn{4}{l}{\textbf{\textcolor{myCOLOR}{\large MODULE 2E: Projects}}} \\
\hline
Thur &  Apr 30 & Final project workshop & \\
Thurs & May 7 & Project Submission Deadline & \\
\hline
\hline 
\end{tabularx}
Note: The class schedule is tentative; it is subject to change as the course progresses.
\end{center}

%\vspace{1cm}
%\section{External Resources}
%
%A list of external resources on diffusion modeling will be collected and provided as they arise.   %click  \href{https://github.com/mikewojnowicz/csci246_fall2025/blob/main/LINKS.md}{here}.  This list will be updated throughout the semester.

\vspace{0.4cm}
%\newpage
%\makeFullPage
\bibliography{references.bib}
\bibliographystyle{unsrt}


\end{document} 


