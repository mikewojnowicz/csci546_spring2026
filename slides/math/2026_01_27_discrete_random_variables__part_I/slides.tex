\documentclass[10pt]{beamer}



%%%
% PREAMBLE FOR THIS DOC 
%%%
%https://tex.stackexchange.com/questions/68821/is-it-possible-to-create-a-latex-preamble-header
\usepackage{/Users/miw267/Repos/csci246_spring2025/slides/preambles/beamer_preamble_for_CSCI246}

\usepackage{wrapfig}  


%%% TRY TO RESHOW TOC AT EACH SECTION START (with current section highlighted)
% Reference: https://tex.stackexchange.com/questions/280436/how-to-highlight-a-specific-section-in-beamer-toc
\newcommand\tocforsect[2]{%
  \begingroup
  \edef\safesection{\thesection}
  \setcounter{section}{#1}
  \tableofcontents[#2,currentsection]
  \setcounter{section}{\safesection}
  \endgroup
}


\usepackage[normalem]{ulem} % for strikeout (\sout)

%%%% HERES HOW TO DO IT CORRECTLY
% FIRST IN .STY FILE, DO
%\usetheme[sectionpage=none]{metropolis}
% THEN AT EACH SECTION DO
%\begin{frame}{Outline}
%  \tableofcontents[currentsection]	
%\end{frame}



%\setbeamertemplate{navigation symbols}{}
%\setbeamertemplate{footline}[frame number]{}


%%%
% DOCUMENT
%%%

\begin{document}

%\maketitle

%% Title page frame
%\begin{frame}
%    \titlepage 
%\end{frame}



\title{01/27/2026: Discrete Random Variables (Part 1)}
\author{CSCI 546: Diffusion Models}
\date{Textbook reference: Sec 3.1-3.5}

\begin{frame}
    \titlepage 
\end{frame}

\begin{frame}
%\begin{myyellowbox}[title=Opening Discussion ($\approx$ 5 mins)]
%Find a partner and discuss these questions.
%\begin{enumerate}
%	\item What is your early reflection on the class?
%		\begin{itemize}
%		\item[a)] goal
%		\item[b)] hope
%		\item[c)] fear
%		\item[d)] like
%		\item[e)] dislike
%		\end{itemize}
%	\item Reflect on your experience reading the textbook.
%\end{enumerate}
%\end{myyellowbox}
%\vfill 
%\pause 
\begin{mygreenbox}[title=Announcement (Sign-in Sheet)]
Please sign the sign-in sheet.
\end{mygreenbox}
\vfill 
\pause 
\begin{myredbox}[title=\text{Announcement (Office Hours)}]
My office hours this Thursday need to be changed due to a grant meeting.  If you would like to meet on Thursday, please send me an email to set up a time.
\end{myredbox}
\vfill 
\pause 
\begin{myyellowbox}[title=\text{Announcement (Group exercises)}]
Group exercises will be posted to the course repo after class. Please continue working on what you don't finish in class.  
\end{myyellowbox}
\end{frame}

\begin{frame}[standout]
Bayes Law Practice
\end{frame}


\begin{frame}[standout]
Review Problem Set \#3	
\end{frame}


\begin{frame}{Binomial Distribution}

The \textbf{Binomial distribution} with parameters $n,p$ is the discrete probability distribution of the number of successes in a sequence of $n$ independent experiments, each asking a yes-no question, whose outcome is yes (or ``success'') with probability $p$ and no (or ``failure'') with probability $1-p$.
\pause 
\begin{figure}
\includegraphics[width=.5\textwidth]{images/binomial_distribution}
\end{figure}
\pause 
\[ P(X=k) = \binom{n}{k} p^{k} (1-p)^{n-k} \]

\end{frame}


\begin{frame}{Poisson Distribution}

The \textbf{Poisson distribution} with rate $\lambda$ is a discrete probability distribution that calculates the likelihood of a certain number of events occurring within a fixed interval of time, assuming the events occur independently.
\pause 
\begin{figure}
\includegraphics[width=.6\textwidth]{images/poisson_distribution}
\end{figure}
\pause 
\[ P(X=k) = \frac{\lambda^k}{k!} e^{-\lambda} \]

	
\end{frame}


\begin{frame}{Random Groups}

\begin{columns}
\begin{column}{0.33\textwidth}
Aubrey Williams: 5 \\ 
Austin Barton : 2 \\ 
Blake Sigmundstad: 1 \\ 
Diego Moylan: 6 \\ 
Dillon Shaffer: 3 \\ 
Felicia Jayasaputra: 1 \\ 
Ismoiljon Muzaffarov: 7 \\\end{column}
\begin{column}{0.33\textwidth}
Jacob Tanner: 4 \\ 
Josh Stoneback: 2 \\ 
Joshua Bowen: 3 \\ 
Joshua Calwell: 5 \\ 
Laura Banaszewski: 1 \\ 
Lina Hammel: 8 \\ 
Logan Racz: 8 \\\end{column}
\begin{column}{0.33\textwidth}
Matt Hall: 7 \\ 
Micah Miller: 4 \\ 
Mike Kadoshnikov: 2 \\ 
Owen Cool: 6 \\ 
Racquel Bowen: 4 \\ 
Samuel Mocabee: 5 \\ 
Tatiana Kirillova: 3 \\\end{column}
\end{columns}

\end{frame}



\begin{frame}{Group exercises - Problem Set 4}
\begin{enumerate}
\item (3.2.1) Let $X$ and $Y$ be independent random variables, each taking the values -1 and 1 with probability $\frac{1}{2}$, and let $Z=XY$. Show that $X,Y$ and $Z$ are pairwise independent. Are they independent?
\item (3.5.2) In your pocket is a random number $N$ of coins, where $N$ has the Poisson distribution with parameter $\lambda$.  You toss each coin once, with heads showing with probability $p$ each time.  Show that the total number of heads has the Poisson distribution with parameter $\lambda p$.
\end{enumerate}   
\end{frame}


\end{document}
